%
% This file by Gareth Moore 12/2/02
%

\newpage
\mysect{LNewMap}{LNewMap}

\mysubsect{Function}{Function}

\index{ladapt@\htool{LNewMap}|(}
This tool will create an empty word map suitable for use with
\htool{LGPrep}. 

\mysubsect{Use}{LNewMap-Use}

\htool{LNewMap} is invoked by the command line
\begin{verbatim}
   LNewMap [options] name mapfn
\end{verbatim}
A new word map is created with the file name `mapfn', with its
constituent {\tt Name} header set to the text passed in `name'.
It also creates default {\tt SeqNo}, {\tt Entries}, {\tt EscMode} and {\tt Fields}
headers in the file.  The contents of the {\tt EscMode} header may be
altered from the default of {\tt RAW} using the {\tt -e} option, whilst
the {\tt Fields} header contains {\tt ID} but may be added to using the {\tt
-f} option.

The allowable options to \htool{LNewMap} are therefore
\begin{optlist}
  \ttitem{-e esc} Change the contents of the {\tt EscMode} header to
      {\tt esc}. Default is {\tt RAW}.

  \ttitem{-f fld} Add the field {\tt fld} to the {\tt Fields} header.

\end{optlist}
\stdopts{LNewMap}

\mysubsect{Tracing}{LNewMap-Tracing}

\htool{LNewMap} supports the following trace options where each trace flag is 
given using an octal base
\begin{optlist}
  \ttitem{00001}  basic progress reporting
\end{optlist}
Trace flags are set using the \texttt{-T} option or the \texttt{TRACE}
configuration variable.
\index{ladapt@\htool{LNewMap}|)}
