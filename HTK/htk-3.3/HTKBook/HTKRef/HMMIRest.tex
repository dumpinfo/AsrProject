%/* ----------------------------------------------------------- */
%/*                                                             */
%/*                          ___                                */
%/*                       |_| | |_/   SPEECH                    */
%/*                       | | | | \   RECOGNITION               */
%/*                       =========   SOFTWARE                  */ 
%/*                                                             */
%/*                                                             */
%/* ----------------------------------------------------------- */
%/* developed at:                                               */
%/*                                                             */
%/*      Speech Vision and Robotics group                       */
%/*      Cambridge University Engineering Department            */
%/*      http://svr-www.eng.cam.ac.uk/                          */
%/*                                                             */
%/* ----------------------------------------------------------- */
%/*         Copyright:                                          */
%/*                                                             */
%/*              2002  Cambridge University                     */
%/*                    Engineering Department                   */
%/*                                                             */
%/*   Use of this software is governed by a License Agreement   */
%/*    ** See the file License for the Conditions of Use  **    */
%/*    **     This banner notice must not be removed      **    */
%/*                                                             */
%

% HTKBook - this chapter written by Dan Povey 01/11/02
%

\newpage
\mysect{HMMIRest}{HMMIRest}

\mysubsect{Function}{HMMIRest-Function}

\index{hmmirest@\htool{HMMIRest}|(}

This program is used to perform a single re-estimation of a HMM set using
a discriminative training criterion.  Maximum Mutual Information (MMI),
Minimum Phone Error (MPE) and  Minimum Word Error (MWE) are supported.
Gaussian parameters are re-estimated using the Baum-Welch update.  

Discriminative training needs to be initialised with trained models; these
will generally be trained using MLE.  Also essential are lattices.
Two sets of lattices are needed: a lattice for the correct transcription of
each training file, and a lattice derived from the recognition of each
training file.  These are called the ``numerator'' and ``denominator''
lattices respectively, a reference to their respective positions in the
MMI objective function, which can be expressed as follows:

\newcommand{\num}{{\mathrm{num}}}
\newcommand{\den}{{\mathrm{den}}}
\newcommand{\MMI}{{\mathrm{MMI}}}
\newcommand{\calO}{ {\cal O} }
\newcommand{\calM}{ {\cal M} }

\begin{equation}
    {\cal F}_\MMI(\lambda) = \sum_{r=1}^R \log \frac {  p^\kappa(\calO_r | \calM^\num_r) } { p^\kappa(\calO | \calM^\den_r) },
  \label{eq:mmimodels}
\end{equation}
where $\lambda$ is the HMM parameters,  $\kappa$ is a scale on the likelihoods, $\calO_r$ is the
speech data for the $r$'th training file and $\calM^\num_r$ and $\calM^\den_r$ are the numerator
and denominator models.  

The numerator and denominator lattices need to contain phone alignment information
(the start and end times of each phone) and language model likelihoods.
Phone alignment information can be obtained by using \htool{HDecode} with
the letter \texttt{d} included in the lattice format options (specified to \htool{HDecode} by the
\texttt{-q} command line option).  It is important that the language model
applied to the lattice not be too informative (e.g., use a unigram LM) and that the
pruning beam used to create it be large enough (probably greater than 100).
Typically an initial set of lattices would be created by recognition using
a bigram language model and a pruning beam in the region of 125-200
(e.g. using \htool{HDecode}, and this would then have a unigram language
model applied to it and a pruning beam applied in the region 100-150, using
the tool \htool{HLRescore}.   The phone alignment information
can then be added using \htool{HDecode}.

Having created these lattices using an initial set of models, the
tool \htool{HMMIRest} can be used to re-estimate the HMMs for 
typically 4-8 iterations, using the same set of lattices.   

\htool{HMMIRest} supports multiple mixture Gaussians, tied-mixture
HMMs, multiple data streams, parameter tying within and between models
(but not if means and variances are tied independently), and diagonal
covariance matrices only.


%\htool{HMMIRest} supports variance flooring in the same way as \htool{HERest}.

%It also supports \textit{Single pass retraining}, useful when the parameterisation of
%the front-end (e.g. from MFCC to PLP coefficients) is to be modified.


Like \htool{HERest}, \htool{HMMIRest} includes features to allow parallel
operation where a network of processors is available. When the training set
is large, it can be split into separate chunks that are processed in
parallel on multiple machines/processors, speeding up the training process.
\htool{HMMIRest} can operate in two modes:
\begin{enumerate}
\item Single-pass operation: the data files on the command line are the
  speech files to be used for re-estimation of the HMM set; the re-estimated
  HMM is written by the program to the directory specified by the {\texttt{-M}}
   option.
\item Two-pass operation with data split into \texttt{P} blocks:
   \begin{enumerate}  
     \item First pass: multiple jobs are started with the options \texttt{-p 1}, 
       \texttt{-p 2} $\ldots$ \texttt{-p P} are given; the different subsets of training
     files should be given on each command line or in a file specified by the \texttt{-S}
     option.  Accumulator files \texttt{<dir>/HDR<p>.acc.1},  \texttt{<dir>/HDR<p>.acc.2} 
    and (for MPE) \texttt{<dir>/HDR<p>.acc.3}  are written, where \texttt{<dir>} is
    the directory specified by the $\texttt{-M}$  and \texttt{<p>} is the number specified
     by the \texttt{-p} option.  
     \item Second pass: \htool{HMMIRest} is called with \texttt{-p 0}, and the
      command-line arguments are the accumulator files written by the first pass.
     The re-estimated HMM set is written to the directory specifified by the $\texttt{-M}$ option. 
   \end{enumerate}
\end{enumerate}



\mysubsect{Use}{HMMIRest-Use}
 \label{sec:hmmirestuse}

\htool{HMMIRest} is invoked via the command line
\begin{verbatim}
   HMMIRest [options] hmmList trainFile ...
\end{verbatim}
or alternatively if the training is in parallel mode, the second pass is invoked as
\begin{verbatim}
   HMMIRest [options] hmmList accFile ...
\end{verbatim}

As always, the list of arguments can be stored in a script file (\texttt{-S} option) if required.  

\subsubsection{Typical use}

To give an example of typical use, for single-pass operation the usage might be:
\begin{verbatim}
HMMIRest -A -V -D -H <startdir>/MMF -M <enddir> -C <config> -q <num-lat-dir> 
        -r <den-lat-dir> -S <traindata-list-file> <hmmlist>
\end{verbatim}
For two-pass operation, the first pass would be (for the \texttt{n}'th training file),
\begin{verbatim}
HMMIRest -p <p> -A -V -D -H <startdir>/MMF -M <enddir> -C <config> -q <num-lat-dir>
        -r <den-lat-dir> -S <p'th-traindata-list-file> <hmmlist>
\end{verbatim}
and the second pass would be
\begin{verbatim}
HMMIRest -p 0 -A -V -D -H <startdir>/MMF -M <enddir> -C <config> -q <num-lat-dir> 
   -r <den-lat-dir> <hmmlist> <enddir>/HDR1.acc.1 <enddir>/HDR1.acc.2 
   <enddir>/HDR2.acc.1  <enddir>/HDR2.acc.2 ... <enddir>/HDR<P>.acc.2 
\end{verbatim}
This is for MMI training; for MPE there will be three files \texttt{HDRn.acc.1,HDRn.acc.2,HDRn.acc.3}
per block of training data.  \texttt{ISMOOTHTAU} can also be tuned.

\subsubsection{Typical config variables}

The most important config variables are given as follows for a number of example setups.
MMI training is the default behaviour.
\begin{enumerate}  
   \item I-smoothed MMI training: set \texttt{ISMOOTHTAU=100}
   \item MPE training (``approximate-accuracy'', which is default kind of MPE): set \texttt{MPE=TRUE}, \texttt{ISMOOTHTAU=50}
   \item ``Exact'' MPE training: set \texttt{MPE=TRUE}, \texttt{EXACTCORRECTNESS=TRUE}, \texttt{INSCORRECTNESS=-0.9}
        (this  may have to be tuned in the range -0.8 to -1; it affects the testing 
        insertion/deletion ratio)
   \item ``Approximate-error'' MPE training (recommended): set \texttt{MPE=TRUE}, \texttt{CALCASERROR=TRUE}, \texttt{INSCORRECTNESS=-0.9}
        (this  may have to be tuned in the range -0.8 to -1; it affects the testing 
        insertion/deletion ratio)
   \item MWE training: set \texttt{MWE=TRUE}, \texttt{ISMOOTHTAU=25}
\end{enumerate}

In all cases set \texttt{LATPROBSCALE=<f>}, where \texttt{<f>} will
normally be in the range 1/10 to 1/20, typically the inverse of the
language model scale.



\subsubsection{Storage of lattices}

If there are great many training files, the directories specified for the numerator and denominator lattice
can contain subdirectories containing the actual lattices.  The name of the subdirectory required can
be extracted from the filenames by adding config variables for instance as follows:
\begin{verbatim}
LATMASKNUM = */%%%?????.???
LATMASKDEN = */%%%?????.???
\end{verbatim}
which would convert a fileneme \texttt{foo/bar12345.lat} to \texttt{<dir>/bar/bar12345.lat},
where \texttt{<dir>} would be the directory specified by the \texttt{-q} or \texttt{-r} option.
The \texttt{\%}'s are converted to the directory name and other characters are discarded.
If lattices are gzipped in order to save space, they can be read in by adding the 
config
\begin{verbatim}
HNETFILTER='gunzip -c < $.gz'
\end{verbatim}


\mysubsect{Command-line options}{HMMIRest-options}



The full list of the options accepted by \htool{HMMIRest} is as follows.
Options not ever needed for normal operation are marked ``obscure''.

\begin{optlist}

  \ttitem{-d dir} 
      (obscure)
      Normally \htool{HMMIRest} looks for HMM definitions
       (not already loaded via MMF files) 
      in the current directory.  This option tells \htool{HMMIRest} to look in
      the directory {\tt dir} to find them.

  \ttitem{-g~~~~} (obscure) Maximum Likelihood (ML) mode-- Maximum Likelihood lattice-based
   estimation using only the numerator (correct-sentence) lattices.

  \ttitem{-l~~~~} (obscure) Maximum number of sentences to use (useful only for troubleshooting)

  \ttitem{-o ext} (obscure) This causes the file name extensions of the
      original models (if any) to be replaced by {\tt ext}.

  \ttitem{-p N~}  Set parallel mode.  1, 2 $\ldots$ N for a forward-backward
     pass on data split into N blocks.  0 for re-estimating using accumulator
    files that this will write.  Default (-1) is single-pass operation.

  \ttitem{-q dir}   Directory to look for numerator lattice files.  These files
      will have a filename the same as the speech file, but with extension
      \texttt{lat} (or as given by \texttt{-Q} option).

  \ttitem{-qp s}   Path pattern for the extended path to look for numerator lattice files. 
      The matched string will be spliced to the end of the directory string specified by
      option `-q' for a deeper path.

  \ttitem{-r dir}   Directory to look for denominator lattice files.

  \ttitem{-rp s}   Path pattern for the extended path to look for denominator lattice files.
     The matched string will be spliced to the end of the directory string specified by
      option `-r' for a deeper path.

  \ttitem{-s file}  File to write HMM statistics.

  \ttitem{-twodatafiles} (obscure) Expect two of each data file for single pass retraining.  
    This works as for HERest; command line contains alternating alignment and update
    files.

  \ttitem{-u flags} By default, \htool{HMMIRest} updates all of the HMM parameters,
      that is, means, variances, mixture weights and 
      transition probabilies. This 
      option causes just the parameters indicated by the {\tt flags}
      argument to be updated, this argument is a string containing one
      or more of the letters {\tt m} (mean), {\tt v} (variance) ,
      {\tt t} (transition) and {\tt w} (mixture weight).  The 
      presence of a letter enables
      the updating of the corresponding parameter set.

  \ttitem{-umle flags} (obscure) A format as in \texttt{-u}; directs that the specified parameters
     be updated using the ML criterion (for MMI operation only, not MPE/MWE).

  \ttitem{-w floor} (obscure) Set the mixture weight floor as a multiple
    of \texttt{MINMIX=1.0e-5}.  Default is 2.

  \ttitem{-x ext}  By default, \htool{HMMIRest} expects a HMM definition for 
      the label X to be stored in a file called {\tt X}.  This
      option causes \htool{HMMIRest} to look for the HMM definition in the
      file {\tt X.ext}.
\stdoptB
\stdoptF
\stdoptH
\ttitem{-Hprior MMF}  Load HMM macro file {\tt MMF}, to be used as prior model in adaptation.  Use with config variables {\tt PRIORTAU} etc.  
\stdoptI
\stdoptM

 \ttitem{-Q ext} Set the lattice file extension to {\tt ext}
\end{optlist}

\stdopts{HMMIRest}

\mysubsect{Tracing}{HMMIRest-Tracing}


The command-line trace option can only be set to 0 (trace off)
or 1 (trace on), which is the default.  Tracing behaviour can
be altered by the {\tt TRACE} configuration variables in the modules \htool{HArc}
and \htool{HFBLat}.

\mysubsect{Configuration variables}{Configuration variables for discriminative training}

The following configuration variables in various modules control the
discriminative training setup.  Other than the ones mentioned in
Section~\ref{sec:hmmirestuse}, these are all set to sensible default
values.  

\subsubsection{Configuration variables in \htool{HMMIRest}}


\begin{itemize}
\item[{\tt TRACE}] - level of trace output, default is 1
\item[{\tt CALCASERROR}] - if TRUE, the code uses ``approximate error'' MPE (as opposed to ``approximate-accuracy'') 
\item[{\tt CW}] - smoothing constant for mixture weight updates, default is 1.0
\item[{\tt CT}] - smoothing constant for transition updates, default is 1.0
\item[{\tt C}] - sets {\tt CW} and {\tt CT} to float value provided
\item[{\tt E}] - smoothing constant for Gaussian updates, default is 2.0
\item[{\tt HCRIT}] - for H-criterion (in MMI case), default is 1.0
\item[{\tt ISMOOTHTAU}] - constant for I-smoothing (typically in the range 25-200).  Default is 0.0
\item[{\tt ISMOOTHTAUW}] - constant for smoothing of mixture weights (e.g. 10).  Default is 0.0
\item[{\tt ISMOOTHTAUT}] - constant for smoothing of transitions (e.g. 10).   Default is 0.0
\item[{\tt LATMASKNUM}]  - pattern-matching string  as described in 
  ~Section~\ref{sec:hmmirestuse} 
\item[{\tt LATMASKDEN}]  - pattern-matching string  as described in 
  ~Section~\ref{sec:hmmirestuse} 
\item[{\tt MINOCC}] - sets minimum occupation count for Gaussians to be updated, default is 10
\item[{\tt MINOCCTRANS}] - sets minimum occupation count for transiton matrix rows to be updated, default is 10
\item[{\tt MINOCCWEIGHTS}] - sets minimum occupation count for Gaussian mixtures be updated, 
         default is 10
\item[{\tt MIXWEIGHTFLOOR}] - floor on mixture weights, as multiple of {\tt MINMIX}. Default is 2.0
\item[{\tt MLE}] - implement MLE.  Default is {\tt FALSE}
\item[{\tt MPE}] - implement MPE.  Default is {\tt FALSE}
\item[{\tt MWE}] - implement MWE.  Default is {\tt FALSE}
\item[{\tt PRIORTAU}] - relates to Gaussian updates using a prior HMM set for an MMI/MPE version of MAP for e.g. gender, speaker or task
     adaptation.  Use this with {\tt -Hprior} option; suitable value might be 10 to 20.  Default is zero.
\item[{\tt PRIORK}]  - relates to Gaussian MMI/MPE-MAP updates.  To vary between zero if ML (rather than discriminatively) adapted system is better than
                original unadapted system, and 1 if vice versa.  Zero (default) is generally best. 
\item[{\tt PRIORTAUT}] - relates to prior on transition updates, with {\tt -Hprior} option.  Suitable value might be in range 1 to 10.  Default is zero.
\item[{\tt PRIORTAUW}] - relates to prior on weight updates, with {\tt -Hprior} option.  Suitable value might be in range 1 to 10.  Default is zero.
\item[{\tt SAVEBINARY}] - save HMMs in binary format Default is {\tt FALSE}
\item[{\tt VARFLOORPERCENTILE}] - set this e.g. to 20 to floor variances to a
   particular percentile of the distribution. 
\end{itemize}


\subsubsection{Configuration variables in \htool{HArc}}

\begin{itemize}
\item[\texttt{TRACE}] - set it to 1 (default) for basic trace, 3 for debugging output
\item[\texttt{FRAMEDUR}] - if frame duration differs from 0.01sec, set it here (important)
\item[\texttt{LMSCALE}] - overrides language model scale present in the lattice
\item[\texttt{WDPEN}] - overrides word insertion penalty present in the lattice
\end{itemize}

\subsubsection{Configuration variables in \htool{HFBLat}}

\begin{itemize}
\item[\texttt{TRACE}] - Set it to 1 for basic trace, can be used in normal operation
\item[\texttt{LATPROBSCALE}] - Scale on lattice-arc acoustic likelihoods \& language model likelihoods, 
  should normally be set to inverse of normal language model scale (e.g. 1/15)
\item[\texttt{EXACTCORRECTNESS}] - If set to {\tt TRUE}, this implements MPE using an exact
  formulation for calculating correctness of paths
\item[\texttt{MINFORPROB}] - Log likelihood threshold below which no stats are accumulated. Default is 10
\item[\texttt{PROBSCALE}] - Scale on state output likelihoods, not used in normal operation
\item[\texttt{LANGPROBSCALE}] - Scale on language model likelihoods, not used in normal operation
\item[\texttt{PHNINSPEN}] - Phone insertion penalty, can be used as an alternative to or 
   interpolated with a normal language model.  Not normally used
\item[\texttt{NOSILENCE}] - If set to {\tt TRUE}, this omits silence from the reference transcription
    when calculating approximated correctness in MPE training.  Default is {\tt FALSE}
\item[\texttt{QUINPHONE}] - If set to {\tt TRUE}, this activates code designed for MPE
  training with quinphone models.  The code is reliant on certain quinphone naming conventions,
  see comments in code for explanation
\item[\texttt{USECONTEXT}] - If set to {\tt TRUE}, compare phones with context for purposes
  of calculating correctness.  Not generally useful
\item[\texttt{INSCORRECTNESS}] - The correctness assigned to an insertion, default is -1.
  Values such as -0.9 may be useful for the exact version of MPE.  This may help
   bring the testing insertion/deletion ratio to a normal value 
\end{itemize}

The configuration variable {\tt MWE} is read here as well as in
\htool{HMMIRest}. 


\subsubsection{Configuration variables in \htool{HExactMPE}}

These configuration variables relate to the exact implementation of MPE (this is
used if \\{\tt EXACTCORRECTNESS=TRUE}).
\begin{itemize}
\item[\texttt{PHONEBEAM}] - A beam width relating to positions in the phone sausage.  Default is 4
\item[\texttt{EXACTCORRPRUNE}] - A pruning beam on the likelihoods, default is 8.5  as opposed to the 
 wider beam of \texttt{MINFORPROB}=10 used in \htool{HFBLat}
\item[\texttt{SUBCORRECTNESS}] - Negated error due to a substitution in Exact MPE, default is -1

\end{itemize}
In addition, a number of configuration variables read in \htool{HFwdBkwdLat} are also read in 
this module.



\index{hmmirest@\htool{HMMIRest}|)}


%%% Local Variables: 
%%% mode: latex
%%% TeX-master: "../htkbook"
%%% End: 
